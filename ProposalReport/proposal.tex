%% You can use this file to create your answer for Exercise 1.  
%% Fill in the places labeled by comments.
%% Generate a PDF document by with the command `pdflatex ex1'.

\documentclass[11pt]{article}

\usepackage{fullpage}
\usepackage{amsmath,amsopn}
\usepackage{graphicx}
\usepackage{color}
\usepackage{verbatim}
\usepackage{setspace}
\usepackage{gensymb}
\usepackage{float}
\usepackage{pgfplots}
\pgfplotsset{compat=1.16}
\usepackage[normalem]{ulem}
%\usepackage[dvips,bookmarks=false,colorlinks,urlcolor=blue,pdftitle={
\usepackage{caption}
\usepackage{subcaption}
\usepackage{amssymb}
\usepackage{amsfonts}
\usepackage{amsthm}
\usepackage{comment}
\usepackage{times}
\usepackage{listings}
\usepackage{enumerate}
\usepackage{courier}
\usepackage{hyperref}
\usepackage{xcolor}
\usepackage{verbatimbox}
\usepackage{tikz}
\usepackage{tikzscale}
\usepgfplotslibrary{groupplots}
\usepackage{float}

\def\vO{{\bf O}}
\def\vP{{\bf P}}
\def\vp{{\bf p}}
\def\vx{{\bf x}}
\def\vl{{\bf l}}
\def\mS{{\bf S}}
\def\mT{{\bf T}}
\def\mH{{\bf H}}
\def\mA{{\bf A}}
\def\tx{{\tilde{\bf x}}}
\def\ta{{\tilde{\bf a}}}
\def\tb{{\tilde{\bf b}}}
\def\tc{{\tilde{\bf c}}}
\def\hn{{\bf \hat{n}}}
\def\hv{{\bf \hat{v}}}
\def\hh{{\bf \hat{h}}}
\def\vh{{\bf h}}
\def\vs{{\bf s}}
\def\hs{{\bf \hat{s}}}
\newcommand{\R}{\mathbb{R}}
\newcommand{\ud}{\,\mathrm{d}}

%% Values that are specific to a particular term
\newcommand{\thisterm}{Fall 2023}

\newcommand{\dateassigned}{Wed., Sep. 13}

%% Printed form of home page that students should use
\newcommand{\visiblecoursehome}{http://www.cs.cmu.edu/\textasciitilde{}418}

%% Link to home page that will stay valid
\newcommand{\actualcoursehome}{http://www.cs.cmu.edu/afs/cs.cmu.edu/academic/class/15418-s23/www}

\newcommand{\datedueregistered}{Wed.,~Sep.~27}

%% Page layout
\oddsidemargin 0pt
\evensidemargin 0pt
\textheight 600pt
\textwidth 469pt
\setlength{\parindent}{0em}
\setlength{\parskip}{1ex}

%% Colored hyperlink 
\newcommand{\cref}[2]{\href{#1}{\color{blue}#2}}
\newcommand{\todo}[1]{[\textcolor{red}{\textit{TODO: }{#1}}]}

%% Customization to listing
\lstset{basicstyle=\ttfamily,language=C++,morekeywords={uniform,foreach}}

%% Enumerate environment with alphabetic labels
\newenvironment{choice}{\begin{enumerate}[A.]}{\end{enumerate}}
%% Environment for supplying answers to problem
\newenvironment{answer}{\begin{minipage}[c][1.0in]{\textwidth}}{\end{minipage}}
\newenvironment{answer2}{\begin{minipage}[c][1.0in]{\textwidth}}{\end{minipage}}

\begin{document}
\begin{center}
\LARGE
15-418/618 \thisterm{} Final Project Proposal
\\ 
PROJECT TITLE
\end{center}
\begin{flushright}
{\large\bf Full Names: Sarah Di, Jinsol Park\makebox[2in][l]{
%% Put your name on the next line

}}

{\large\bf Andrew IDs: sarahdi, jinsolp\makebox[2in][l]{\tt
%% Put your Andrew ID on the next line

}}
\end{flushright}

{\large\bf Project Page URL: \url{https://github.com/disarah/15618\_Project}\makebox[2in][l]{
%% Put your name on the next line

}}

Highlight attention matters in performance
execution time of attention vs execution of other layers
attention dominates computation
optimizing important components

bottleneck of performance

\section{Summary}
(2-3 sentences) Describe what you plan to do and what parallel systems you will be working with. Example one-liners include (you should add a bit more detail):

• We are going to implement an optimized Smoothed Particle Hydrodynamics fluid solver on the NVIDIA GPUs in the lab.
• We are going port the Go runtime to Blacklight.

• We are going to create optimized implementations of sparse-matrix multiplication on both GPU and multi-core CPU platforms, and perform a detailed analysis of both systems’ performance characteristics.

• We are going to back-engineer the unpublished machine specifications of the GPU in the tablet my partner just purchased.

• We are going to implement two possible algorithms for a real-time computer vision application on a mobile device and measure their energy consumption in the lab.

\section{Background}
If your project involves accelerating a compute-intensive application, describe the application or piece of the application you are going to implement in more detail. This description need only be a few paragraphs. It might be helpful to include a block diagram or pseudocode of the basic idea. An important detail is what aspects of the problem might benefit from parallelism and why?

\section{The Challenge}
Describe why the problem is challenging. What aspects of the problem might make it difficult to parallelize? In other words, what to you hope to learn by doing the project?

• Describe the workload: what are the dependencies, what are its memory access characteristics? (is there locality? is there a high communication to computation ratio?), is there divergent execution?

• Describe constraints: What are the properties of the system that make mapping the workload to it challenging?

\section{Resources}
Describe the resources (type of computers, starter code, etc.) you will use. What code base will you start from? Are you starting from scratch or using an existing piece of code? Is there a book or paper that you are using as a reference (if so, provide a citation)? Are there any other resources you need, but haven’t figured out how to obtain yet? Could you benefit from access to any special machines?

\section{Goals and Deliverables}
Describe the deliverables or goals of your project. This is by far the most important section of the proposal!

• Separate your goals into what you PLAN TO ACHIEVE (what you believe you must get done to have a successful project and get the grade you expect) and an extra goal or two that you HOPE TO ACHIEVE if the project goes really well and you get ahead of schedule, as well as goals in case the work goes more slowly. It may not be possible to state precise performance goals at this time, but we encourage you be as precise as possible. If you do state a goal, give some justification of why you think you can achieve it. (e.g., I hope to speed up my starter code 10x, because if I did it would run in real-time.)

• If applicable, describe the demo you plan to show at the poster session (Will it be an interactive demo? Will you show an output of the program that is really neat? Will you show speedup graphs?). Specifically, what will you show us that will demonstrate you did a good job?

• If your project is an analysis project, what are you hoping to learn about the workload or system being studied? What question(s) do you plan to answer in your analysis?

• Systems project proposals should describe what the system will be capable of and what performance is hoped to be achieved.

\section{Platform Choice}
Describe why the platform (computer and/or language) you have chosen is a good one for your needs. Why does it make sense to use this parallel system for the workload you have chosen?

\section{Schedule}
Produce a schedule for your project. Your schedule should have at least one item to do per week. List what you plan to get done each week from now until the parallelism competition in order to meet your project goals. Keep in mind that due to other classes, you’ll have more time to work some weeks than others (work that into the schedule). You will need to re-evaluate your progress at the end of each week and update this schedule accordingly. Note the intermediate milestone deadline is April 19. In your schedule we encourage you to be precise as precise as possible. It’s often helpful to work backward in time from your deliverables and goals, writing down all the little things you’ll need to do (establish the dependencies).

Submitting your proposal writeup: Please post your writeup on your project web page. (Note that you will also be posting updates and results to this web page, so your proposal should be a link on your web page, not the entire web page itself.) In addition, please submit your writeup to Gradescope (as a group submission). As noted earlier, that writeup should include a URL for your project web page. (Note: the Gradescope submission should be a PDF of the full proposal; not just a link to the proposal on your project web site.)

\end{document}