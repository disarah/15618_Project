\begin{table}[t]
\small
\centering
\renewcommand{\arraystretch}{1.2}
\setlength{\tabcolsep}{7.5pt}
% =====================================================================
%                           Window sparse
% =====================================================================
\begin{subtable}{\linewidth}\centering
\begin{tabular}{||c|c||}
\hline

\texttt{ws} & Total \\ [0.5ex] 

\hline\hline

16 & 58.11 \\
32 & 97.60 \\
64 & 184.32 \\


\hline
\end{tabular}
\caption{Window-sparse attention on GPU ablation. \texttt{ws} is window size.}
\end{subtable}
% =====================================================================
%                           Random sparse
% =====================================================================
\begin{subtable}{\linewidth}\centering
\begin{tabular}{||c|c||}
\hline

\texttt{RF} & Total \\ [0.5ex] 

\hline\hline

0.02 & 258.40 \\
0.1 & 1337.98 \\
0.2 & 2708.16 \\
0.4 & 5350.53 \\
0.6 & 8139.29 \\


\hline
\end{tabular}
\caption{Random-sparse attention on GPU ablation. \texttt{RF} is the random fraction.}
\end{subtable}
% =====================================================================
%                           Global sparse
% =====================================================================
\begin{subtable}{\linewidth}\centering
\begin{tabular}{||c|c||}
\hline

ID & Total \\

\hline\hline
16 & 344.06 \\
32 & 518.14 \\
64 & 685.66 \\


\hline
\end{tabular}
\caption{Global-sparse attention on GPU ablation. \texttt{ID} is the number of global indices.}
\end{subtable}
\caption{Ablations. Total is the total execution time for all kernels (DDS, softmax, SDD). Experiments are done on sequence length \texttt{N = 1024}.}
\label{table:ablation}
\end{table}